% gjr.tex
\documentclass[11pt]{article}
\usepackage{hyperref}
\usepackage[utf8]{inputenc}
\usepackage{graphicx} % support the \includegraphics command and optio
\usepackage[fleqn]{amsmath}
\usepackage{amssymb}
\usepackage{amsfonts}
\usepackage{amsthm}

\newcommand{\Var}{\mathop{\rm{Var}}}
\newcommand{\Cov}{\mathop{\rm{Cov}}}
\theoremstyle{definition}
\newtheorem*{example}{}

% \usepackage[parfill]{parskip} % Activate to begin paragraphs with an empty line rather than an indent

\title{Normal Jarrow-Rudd}
\author{Keith A. Lewis}
\date{\today}

\begin{document}
\maketitle
\begin{abstract}
The Black-Scholes/Merton model has been Nobel prize winning successful,
but parameterizing models that fit volatility smiles at even a single
maturity for European options has been a remarkably
intractable problem.

Instead of working with the difference of the cumulants from a lognormal
distribution as in Jarrow-Rudd \cite{JarRud1982} we work with the
difference of the cumulants from a normal distribution.  This is more
natural since all but the first two cumulants of a normal distribution
are zero. For other distributions, the third and fourth cumulant are closely
related to the skew and kurtosis.

The mathematics involved is fairly elementary and a software
implementation is available for you to try out your ideas
on how this can be used.
\end{abstract}

\section{Outline}
The forward value of an European put option having strike \(k\)
and forward $F$ at expiration is
\[
E\max\{k - F, 0\} = E(k - F)1(F \le k) = kP(F\le k) - fP^*(F\le k)
\]
where $f = EF$ and $dP^*/dP = F/f$
is the {\em Esscher transform}\cite{Ess1932}.
We derive formulas for
\(P(F\le k)\) and \(P^*(F\le k)\) that can be computed
efficiently when $\log F$ is approximately normal.

This generalizes the Black formula
\(E\max\{k - F,0\} = k\Phi(-d_2) - f\Phi(-d_1)\)
where \(F = fe^{-\sigma^2t/2 + \sigma B_t}\), \((B_t)\) is
standard Brownian motion, \(\Phi\) is the standard normal cumulative
distribution function, \(d_1 = (\sigma^2 t/2
+ \log f/k)/\sigma\sqrt{t}\) and \(d_2 = d_1 - \sigma\sqrt{t}\).

In section 2 we recall some basic facts about cumulants.  Section 3 uses
an Edgeworth expansion and Bell polynomials to show their relationship
to moments.
The third section describes how Hermite polynomials
give an explicit formula for the cumulative
distribution function of the perturbed distribution.  Option valuation
is considered in section 4 where we show how the Esscher transform can be
calculated using the method in section 3. The last section
collects some general remarks.

\section{Cumulants}

The {\em cumulant} of the random variable \(X\)
is \(\kappa(s) = \log Ee^{sX}\).
The {\em cumulants}, \((\kappa_n)\),
are defined by
\[
\kappa(s) = \sum_{n=1}^\infty \kappa_n \frac{s^n}{n!}
\]
Since \((d/ds)^n\kappa(s)|_{s = 0} = \kappa_n\) it is easy to
work out that
\(\kappa_1 = EX\) and \(\kappa_2 = \Var(X)\). Higher order
cumulants are less intuitive but the third and fourth are
related to skew and kurtosis.

The cumulants of a random variable plus a constant are the 
same except the first cumulant is increased by the constant.
More generally, the cumulants of the sum of two independent 
random variables are the sums of their cumulants.
They scale homogeneously, the \(n\)-th cumulant of a constant
times a random variable is
\(\kappa_n(cX) = c^n\kappa_n(X)\).

The relationship between cumulants and moments, \(m_n = EX^n\),
involves Bell
polynomials\cite{Bel1934}.
\[
Ee^{sX} = \sum_{n=0}^\infty m_n \frac{s^n}{n!}
 = \exp(\sum_{n=1}^\infty \kappa_n \frac{s^n}{n!})
= \sum_{n=0}^\infty B_n(\kappa_1,\dots,\kappa_n) \frac{s^n}{n!}
\]
where \(B_n(\kappa_1,\dots,\kappa_n)\) is the \(n\)-th complete
Bell polynomial.
This is just a special case of the
Fa\`a di Bruno formula first proved by Louis Fran\c{c}ois Antoine
Arborgast in 1800\cite{Arb1800}.
Bell polynomials satisfy the recurrence \cite{Com1974} \(B_0 = 1\) and
\[
B_{n+1}(x_1,\dots,x_{n+1}) = \sum_{k=0}^n \binom{n}{k}
B_{n - k}(x_1,\dots, x_{n - k}) x_{k+1}.
\]

\section{Edgeworth Expansion}
Given random
variables \(X\) and \(Y\), we have
\[\log E e^{iuY} - \log E e^{iuX} = \sum_{n=1}^\infty \Delta\kappa_n (iu)^n/n!\]
where \((\Delta\kappa_n)\) are the differences of the cumulants 
of \(X\) and \(Y\), so
\[
Ee^{iuY} = Ee^{iuX}\sum_{n=0}^\infty B_n(\Delta\kappa_1,...,\Delta\kappa_n)(iu)^n/n!.
\]

Let \(F\) and \(G\) be the cumulative distribution functions of
\(X\) and \(Y\) respectively. Furthur assume they have density
functions \(f\) and \(g\).
The Fourier transform, \(\hat{f}(u) = Ee^{iuX}\),
of \(f'\) is \(-iu \hat f(u)\) so
the Fourier transform of the \(n\)-th derivative
\(f^{(n)}\) is \((-iu)^n\hat f(u)\) hence
\[
	g(x) = \sum_{n=0}^\infty (-1)^n B_n(\Delta\kappa_1,...,\Delta\kappa_n)
	f^{(n)}(x)/n!
\]

\subsection{Hermite Polynomials}
The derivatives of the standard normal cumulative distribution 
can be computed using Hermite polynomials\cite{AbrSte1964}
pp. 793--801.
One definition is
\[
H_n(x) = (-1)^n e^{x^2/2}\frac{d^n}{dx^n}e^{-x^2/2}.
\]
They satisfy the recurrence \(H_0(x) = 1\), \(H_1(x) = x\) and
\[
H_{n+1}(x) = xH_n(x) - n H_{n-1}(x).
\]
Note some authors use \(He_n(x)\) instead of \(H_n(x)\).

Integrating both sides of the formula for the density
functions and using Hermite polynomials yields
\[
G(x) = F(x) - \sum_{n=1}^\infty
B_n(\Delta\kappa_1,\dots,\Delta\kappa_n) H_{n-1}(x)
e^{-x^2/2}/n!\sqrt{2\pi}
\]

\section{The Esscher Transform}

If \(X\) is standard normal then
\(P^*(X\le z) = Ee^{-s^2/2 + sX} 1(X\le z)) = P(X + s\le z)
= \Phi(z - s)\)
where we use the fact
\[
Ee^M f(N) = Ee^M Ef(N + \Cov(M,N))
\]
if \(M\) and \(N\) are
jointly normal. Here \(M = -s^2/2 + sX\), \(N = X\), and \(f(x) = 1(x\le z)\).

If \(Y\) is a pertubation of \(X\) then
for \(G(z) = P(Y\le z)\), we need to compute
\[
G^*(z) = P^*(Y\le z) = Ee^{-\kappa(s) + sY}1(Y\le z).
\]
The cumulants of \(Y^*\) can be found from
\begin{align*}
\log E e^{uY^*} &= \log E e^{-\kappa(s) + s Y} e^{uY}\\
&= -\kappa(s) + \kappa(s + u)\\
&= \sum_{n=1}^\infty -\kappa_n \frac{s^n}{n!} 
	+ \kappa_n \frac{(u + s)^n}{n!}\\
&= \sum_{n=1}^\infty \sum_{k=0}^{n-1} \kappa_n \binom{n}{k}
	\frac{s^{n - k}u^k}{n!}\\
&= \sum_{k=1}^\infty \sum_{n=k + 1}^\infty \kappa_n 
	\binom{n}{k}\frac{s^{n - k}u^k}{n!}\\
&= \sum_{k=1}^\infty \sum_{n=k + 1}^\infty \kappa_n \frac{s^{n-k}}{(n-k)!}
	\frac{u^k}{k!}\\
&= \sum_{k=1}^\infty 
	\bigl(\sum_{n=1}^\infty \kappa_{n+k} \frac{s^n}{n!}\bigr)
	\frac{u^k}{k!}\\
\end{align*}
This shows the cumulants of \(Y^*\) are 
\(\kappa^*_n = \sum_{k=1}^\infty \kappa_{n + k}s^k/k!\)
so we can also compute \(G^*(z) = P(Y^* \le z)\) using the same
technique as for \(G\).

If \(G\) is standard normal then all cumulants vanish except \(\kappa_2 = 1\).
The only nonzero cumulants of \(Y^*\) are \(\kappa_1 = s\)
and \(\kappa_2 = 1\). The corresponding complete Bell polynomials
are \(B_n = s^n/n!\) and we see the above formula is just
the Taylor series expansion of \(G(z - s)\).
%
%The Edgeworth series expands the exponential in a power series.
%\[\exp\bigl(\sum_{n=0}^\infty \Delta\kappa_n (iu)^n/n!\bigr)
%= \sum_{k=0}^\infty (\sum_{n=0}^\infty \Delta\kappa_n (iu)^n/n!)^k/k!\]
%

%\subsection{Computations}
%Explicit formulas the first few Bell polynomials:
%\begin{align*}
%B_0 &= 1\\
%B_1 &= x_1\\
%B_2 &= B_1 x_1 + B_0 x_2\\
%&= x_1^2 + x_2\\
%B_3 &= B_2 x_1 + 2B_1 x_2 + B_0 x_3\\
%&= x_1^3 + x_1 x_2 + 2x_1 x_2 + x_3\\
%&= x_1^3 + 3x_1 x_2 + x_3\\
%B_4 &= B_3 x_1 + 3 B_2 x_2 + 3 B_1 x_3 + B_0 x_4\\
%&= (x_1^3 + 3x_1 x_2 + x_3) x_1  + 3 (x_1^2 + x_2)x_2 + 3 x_1 x_3 + x_4\\
%&= x_1^4 + 6x_1^2 x_2 + 4 x_1 x_3 + 3x_2^2 + x_4\\
%\end{align*}
%Explicit formulas for the first few Hermite polynomials:
%\begin{align*}
%H_0 &= 1\\
%H_1 &= x\\
%H_2 &= x^2 - 1\\
%H_3 &= x^3 - x\\
%H_4 &= x^4 - 6x^2 + 3\\
%H_5 &= x^5 - 10x^3 + 15x\\
%H_6 &= x^6 - 15x^4 + 45x^2 - 15\\
%\end{align*}
%
%Assuming \(\kappa_1 = \kappa_2 = 0\) the first few terms of the Edgeworth expansion are:
%\begin{align*}
%G(x) &= \sum_{n=0}^\infty (-1)^n B_n F^{(n)}(x)/n!\\
%&= F(x) - B_1 F'(x) + B_2F''(x)/2 - B_3F^{(3)}(x)/6 + B_4F^{(4)}/24\\
%&= F(x) + (-\kappa_3 (x^2 - 1)/6 - \kappa_4 (x^3-x)/24)e^{-x^2/2}/\sqrt{2\pi}\\
%\end{align*}
%The distribution is unimodal if and only if the second derivative has exactly one root.
%\begin{align*}
%G''(x) &= \sum_{n=0}^\infty (-1)^n B_n F^{(n+1)}(x)/n!\\
%&= F^{(2)}(x) - B_1 F^{(3)}(x) + B_2F^{(4)}(x)/2 - B_3F^{(5)}(x)/6 + B_4F^{(6)}/24\\
%&= (-x - \kappa_3 (x^4 - 6x^2 + 3)/6
%	 - \kappa_4 (x^5 - 10x^3 + 15x)/24)e^{-x^2/2}/\sqrt{2\pi}\\
%\end{align*}
%
\subsection{Examples}
\begin{example}[Normal]
If \(X\) is normal with mean \(\mu\) and variance \(\sigma^2\) then
\(\kappa(s) = \mu s + \sigma^2s^2/2\) so
\(\kappa_1 = \mu\) and \(\kappa_2 = \sigma^2\) are the only non-zero
cumulants. We also have \(\kappa_1^* = \mu + \sigma^2s\)
and \(\kappa_2^* = \kappa_2\).

If the cumulants of
a random variable vanish after some some point, then it must
be normal\cite{Luk1970} (Theorem 7.3.5).
Something to keep in mind with computer
implementations.
\end{example}
\begin{example}[Poisson]
If \(X\) is Poisson
with mean \(\mu\) then \(\kappa(s) = \mu(e^s - 1)\) so
\(\kappa_n = \mu\) for all \(n\) and
\(\kappa_n^* = \mu e^s\).
\end{example}
%\begin{example}[Compound Poisson]
%If \(Y\) is Poisson with mean \(\mu\) and \(Z_j\) are
%independent and identically distributed, define
%\(X = \sum_{j=1}^{Y} Z_j\). The cumulants of \(X\)
%are \(\kappa_n = ?\).
%\end{example}
%\begin{align*}
%Ee^{uX} &= \sum_{k=0}^\infty Ee^{u(Z_1 + \cdots Z_k)}\mu^k/k!\,e^{-\mu}\\
%&= \sum_{k=0}^\infty (Ee^{u Z_1})^k\mu^k/k!\,e^{-\mu}\\
%&= \sum_{k=0}^\infty (Ee^{u Z_1}\mu)^k/k!\,e^{-\mu}\\
%&= e^{\mu Ee^{u Z_1}}\,e^{-\mu}\\
%&= e^{\mu (Ee^{u Z_1} - 1)}\\
%\end{align*}
%Define \(\lambda(u) = \log Ee^{uZ_1}\). Then
%\(\log E e^{uX} = \mu(e^{\lambda(u)} -1)\).
\begin{example}[Exponential]
If \(X\) is exponential with mean \(\mu\) then
\(\kappa(s) = -\log(1 - \mu s)\),
\(\kappa_n = (n - 1)!\mu^n\) and
\(\kappa_n^* = n!\mu^n(1/(1 - \mu s)^{n+1} - 1)\)
% sum_k (n + k - 1)!\mu^(n+k)/k! =  
since \(\sum_{k=0}^\infty (n + k - 1)!x^k/k!
= (d/dx)^n(1 - x)^{-1} = n!/(1 - x)^{n + 1}\).
\end{example}
\begin{example}[Gamma]
If \(X\) is gamma with mean \(\alpha/\beta\) and variance
\(\alpha/\beta^2\) then
\(\kappa(s) = -\alpha\log(1 - s/\beta)\) and
\(\kappa_n = (n - 1)!\alpha/\beta^n\) so
\(\kappa_n^* = (n - 1)!\alpha(\beta - s)^{-n}\).

Note gamma is exponential when
\(\alpha = 1\) and \(\beta = 1/\mu\).
\end{example}
%\begin{align*}
%Ee^{uX} &= \int_0^\infty e^{ux} x^{\alpha-1} e^{-\beta x}
%\frac{\beta^\alpha}{\Gamma(\alpha)}\,dx\\
%&= \int_0^\infty x^{\alpha-1} e^{-(\beta -u) x}
%\frac{\beta^\alpha}{\Gamma(\alpha)}\,dx\\
%&= \frac{\Gamma(\alpha)}{(\beta - u)^\alpha} \frac{\beta^\alpha}{\Gamma(\alpha)}\\
%&= (1 - u/\beta)^{-\alpha}\\
%\end{align*}
%so
%\(
%\log E e^{uX} = \alpha \sum_{n=1}^\infty (u/\beta)^n/n\).

%At the money pricing ???
%%F = fe^{-k(s) + s X}, x \le k(s)/s
%
%\begin{align*}
%&P(F\le f) - P^*(F\le f) \\
%&= P(Y\le \kappa(s)/s) - P^*(Y\le \kappa(s)/s) \\
%&= G(\kappa(s)/s) - G^*(\kappa(s)/s) \\
%&= \sum_{n=1}^\infty (-B_n(\Delta\kappa) + B_n(\Delta\kappa^*))
%    e^{-(\kappa(s)/s)^2/2} H_{n-1}(\kappa(s)/s)/sqrt{2\pi} \\
%& = \Phi(s/2) - \Phi(-s/2) \\
%\end{align*}

%\(\chi^2\) with \(r\) degrees of freedom \(\kappa_n = 2^{n-1}(n-1)!r\).


\subsection{L\'evy Processes}

%Let \(X = aN + bP + c\) where \(N\) is standard normal and \(P\)
%is Poisson with mean \(\mu\).

%\(EX = b\mu + c\) and \(\Var X = a^2 + b^2\). Taking
%\(b = \sqrt{1 - a^2}\)

Kolmogorov's precursor to the L\'evy-Khintchine theorem\cite{Kol1992}
states that if a random variable \(X\) is infinitely divisible
and has finite variance
there exists a number \(\gamma\) and a non-decreasing function
\(G\) defined on the real line such that
\[
\kappa(s) = \log Ee^{sX} = \gamma s + \int_{-\infty}^\infty K_s(x)\,dG(x),
\]
where \(K_s(x) = (e^{sx} - 1 - sx)/x^2 = \sum_{n=2}^\infty s^nx^{n-2}/n!\).
Note the first cumulant of \(X\) is \(\gamma\) and for \(n\ge 2\),
\(\kappa_n = \int_{-\infty}^\infty x^{n-2}\,dG(x)\).

The Hamburger moment problem[cite?] provides the answer to
what the allowable cumulants are: the Hankel matrix
\([\kappa_{i+j}]_{i,j\ge 2}\) must be positive definite.

Since \(K_s(0) = s^2/2\) is the cumulant of the standard normal
distribution and \(a^2K_s(a) + as\) is the cumulant of a
Poisson distrbution having mean \(a\),
infinitely divisible random variables can be
approximated by a normal plus a linear combination of
independent Poisson distributions.

%The K-model takes \(\gamma = 0\) and \(G\) of the form
%\[
%G(x) =
%\begin{cases}
%a e^{x/\alpha} &x < 0\\
%1 - be^{-x/\beta} & x > 0\\
%\end{cases}
%\]
%Note \(G\) jumps by \(1 - a - b\) at the origin. If \(a = b = 0\)
%this reduces to a standard normal distribution.
%
%The cumulants are simple to compute: \(\kappa_1 = 0\), \(\kappa_2 = 1\),
%and \(\kappa_{n+2} = (a(-\alpha)^n + b\beta^n)n!\) for \(n > 1\).

\section{Remarks}
The Gram-Charlier A series expands the quotients of cumulative
distribution functions \(G/F\) using Hermite polynomials, but does not
have asymptotic convergence, whereas the Edgeworth expansion involves
the quotient of characteristic functions \(\hat G/\hat F\) in terms of
cumulants and does have asymptotic convergence, ignoring some dainty
facts \cite{Pet1975}.

If \((X_t)\) is a L\'evy process then \(X_1\) is
infinitely divisible and \(\log Ee^{sX_t} = t\kappa(s)\).
A consequence is that the volatility smile at a single
maturity determines the entire volatility surface, a fact that
may indicate L\'evy processes are not appropriate for
modeling stock prices.

A software implementation in C++ is available at
\url{https://fmsgjr.codeplex.com} and Excel add-ins
at \url{https://xllgjr.codeplex.com} that require
\url{https://xll.codeplex.com}.

\bibliographystyle{amsplain}
\bibliography{njr}

\end{document}
